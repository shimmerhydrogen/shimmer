\section{Network Data Files}
The Shimmer++ input/output is done entirely via network data files (NDFs). The
user prepares NDFs specifying the whole network and the code processes them and
stores simulation solutions in them.
The NDFs of Shimmer++ are based on SQLite, which is the
most widely deployed relational database system. This choice guarantees total
compatibility with any programming language/environment and easy manipulation of
the network data via standard SQL queries. On top of the raw SQLite database, Shimmer++ provides a C++
API and a Matlab API specific for the manipulation of the NDFs.

%%%%%%%%%%%%%%%%%%%%%%%%%%%%%%%%%%%%%%%%%%%%%%%%%%%%%%%%%%%%%%%%%%%%%%%%%%%%%%%%
%%%%%%%%%%%%%%%%%%%%%%%%%%%%%%%%%%%%%%%%%%%%%%%%%%%%%%%%%%%%%%%%%%%%%%%%%%%%%%%%
\subsection{Database schema: stations}
In the following the NDF database schema will be described. The knowledge of
the database schema is needed to directly modify the NDFs via tools like
SQLite Browser (\url{https://sqlitebrowser.org/}) or to implement third party
tools able to read and write Shimmer++ NDFs.

%%%%%%%%%%%%%%%%%%%%%%%%%%%%%%%%%%%%%%%%%%%%%%%%%%%%%%%%%%%%%%%%%%%%%%%%%%%%%%%%
\subsubsection{Station types}
Shimmer++ handles gas networks with different types of stations, as described
in Section~\ref{sec:shimmer_model}. Each station type is assigned a specific
numeric identifier in order to be appropriately handled by the code. Currently
implemented station types are visible in \texttt{network\_elements.hpp} and
summarized in the following code snippet: 

\begin{minted}{c++}
enum class station_type : int {
    ENTRY_P_REG = 1,            /* ReMi w/o backflow */
    ENTRY_L_REG = 2,            /* Injection w/ pressure control */
    EXIT_L_REG  = 3,            /* Consumption point w/o pressure control */
    JUNCTION    = 4,            /* Junction */
    PRIVATE_INLET  = 10,        /* Inlet, internal use only */
    PRIVATE_OUTLET = 11,        /* Outlet, internal use only */
    FICTITIOUS_JUNCTION = 100   /* For quality tracking, should not appear in DB*/
};
\end{minted}

On the NDF side, the table containing the station types is named
\texttt{station\_types} and is specified by the following SQL statement:
\begin{minted}{sql}
create table station_types (
    t_type          INTEGER,        -- Station type, as per enum above
    t_descr         TEXT NOT NULL,  -- Free-form description of the type
    t_limits_table  TEXT,           -- Table with type-specific operational limits
    t_profile_table TEXT,           -- Table with type-specific operational profiles

    PRIMARY KEY(t_type)
);
\end{minted}

This table comes pre-populated to reflect the numeric ids cited above and it
should not be modified by the user. 

\begin{table}[ht]
    \centering
    \begin{tabular}{c|c|c|c}
    \texttt{t\_type} & \texttt{t\_descr} & \texttt{t\_limits\_table} & \texttt{t\_profile\_table} \\
    \hline
    1 & ReMi station w/o backflow & \texttt{limits\_remi\_wo} & \texttt{profiles\_remi\_wo} \\
    2 & Injection station w/ pressure control & \texttt{limits\_injection\_w} & \texttt{profiles\_injection\_w} \\
    3 & Consumption point w/o pressure control & \texttt{limits\_conspoint\_wo} & \texttt{profiles\_conspoint\_wo} \\
    4 & Junction & NULL & NULL \\
    10 & Inlet station - private & NULL & NULL \\
    11 & Outlet station - private & \texttt{limits\_outlet\_priv} & \texttt{profiles\_outlet\_priv}
    \end{tabular}
\end{table}

The columns \texttt{t\_limits\_table} and \texttt{t\_profile\_table} contain the table names for the settings of a specific station type. Should the code be modified to support new station types, this table must be updated accordingly.

%\begin{itemize}   
%\end{itemize}

%%%%%%%%%%%%%%%%%%%%%%%%%%%%%%%%%%%%%%%%%%%%%%%%%%%%%%%%%%%%%%%%%%%%%%%%%%%%%%%%
\subsubsection{Stations}
The table \texttt{stations} contains thr list of all the stations present in a
network. A station is uniquely identified by the field \texttt{s\_number} and has
the following attributes:
\begin{itemize}
    \item \texttt{s\_name}: the name of the station
    \item \texttt{t\_type}: the type of the station, according to the table \texttt{station\_types} 
    \item \texttt{s\_height}: height of the station
    \item \texttt{s\_latitude}: latitude of the station
    \item \texttt{s\_longitude}: longitude of the station
\end{itemize}

The table is fully specified by the following SQL code:
\begin{minted}{sql}
create table stations (
    s_number    INTEGER,
    s_name      TEXT NOT NULL,
    t_type      INTEGER,

    s_height    REAL DEFAULT 0.0 NOT NULL,
    s_latitude  REAL DEFAULT 0.0 NOT NULL,
    s_longitude REAL DEFAULT 0.0 NOT NULL,

    PRIMARY KEY(s_number),
    
    -- The type of the station must be well-defined
    FOREIGN KEY (t_type)
        REFERENCES station_types(t_type),
    
    CHECK(s_number >= 0)
);
\end{minted}

%%%%%%%%%%%%%%%%%%%%%%%%%%%%%%%%%%%%%%%%%%%%%%%%%%%%%%%%%%%%%%%%%%%%%%%%%%%%%%%%
\subsubsection{Limits and profiles: ReMi station}
Each station of type ReMi has an associated set of user-specified limits on pressure
and mass flow rate. Limits are specified per station by NDF entries including:
\begin{itemize}
    \item \texttt{s\_number}:   number of the station
    \item \texttt{lim\_Lmin}:   minimum allowed mass flow rate
    \item \texttt{lim\_Lmax}:   maximum allowed mass flow rate
    \item \texttt{lim\_Pmin}:   minimum allowed pressure
    \item \texttt{lim\_Pmax}:   maximum allowed pressure
\end{itemize}
The table of the limits for ReMi stations is fully specified by the following SQL code:
\begin{minted}{sql}
create table limits_remi_wo (
    s_number    INTEGER UNIQUE,
    lim_Lmin    REAL DEFAULT 0.0 NOT NULL,
    lim_Lmax    REAL DEFAULT 0.0 NOT NULL,
    lim_Pmin    REAL DEFAULT 0.0 NOT NULL,
    lim_Pmax    REAL DEFAULT 0.0 NOT NULL,

    FOREIGN KEY (s_number)
        REFERENCES stations(s_number)
);
\end{minted}
In addition to limits, the NDFs allow to store also profiles for the quantity
controlled by the station. A pressure profile is specified by triples including:
\begin{itemize}
    \item \texttt{s\_number}:   number of the station
    \item \texttt{prf\_time}:   relative time of the sample (seconds)
    \item \texttt{prf\_Pset}:   pressure setpoint at the specified time
\end{itemize}
If a profile for a station includes a single entry, the quantity is held constant
at the specified value for the whole simulation. The table of the profiles for
ReMi stations is fully specified by the following SQL code:
\begin{minted}{sql}
create table profiles_remi_wo (
    s_number    INTEGER,
    prf_time    REAL DEFAULT 0.0 NOT NULL,
    prf_Pset    REAL DEFAULT 0.0 NOT NULL,

    FOREIGN KEY (s_number)
        REFERENCES stations(s_number)
);
\end{minted}

%%%%%%%%%%%%%%%%%%%%%%%%%%%%%%%%%%%%%%%%%%%%%%%%%%%%%%%%%%%%%%%%%%%%%%%%%%%%%%%%
\subsubsection{Limits and profiles: Injection station}
Each station of type Injection has an associated set of user-specified limits on pressure
and mass flow rate. Limits are specified per station by NDF entries including:
\begin{itemize}
    \item \texttt{s\_number}:   number of the station
    \item \texttt{lim\_Lmin}:   minimum allowed mass flow rate
    \item \texttt{lim\_Lmax}:   maximum allowed mass flow rate
    \item \texttt{lim\_Pmin}:   minimum allowed pressure
    \item \texttt{lim\_Pmax}:   maximum allowed pressure
    \item \texttt{parm\_f}:     scale factor named "f" in the slides DENERG CLARIFY
\end{itemize}
The table of the limits for Injection stations is fully specified by the following SQL code:
\begin{minted}{sql}
create table limits_injection_w (
    s_number    INTEGER UNIQUE,
    lim_Lmin    REAL DEFAULT 0.0 NOT NULL,
    lim_Lmax    REAL DEFAULT 0.0 NOT NULL,
    lim_Pmin    REAL DEFAULT 0.0 NOT NULL,
    lim_Pmax    REAL DEFAULT 0.0 NOT NULL,
    parm_f      REAL DEFAULT 1.0 NOT NULL,

    FOREIGN KEY (s_number)
        REFERENCES stations(s_number)
);
\end{minted}
In addition to limits, the NDFs allow to store also profiles for the quantity
controlled by the station. A profile for Injection stations is specified by quadruples including:
\begin{itemize}
    \item \texttt{s\_number}:   number of the station
    \item \texttt{prf\_time}:   relative time of the sample (seconds)
    \item \texttt{prf\_Pset}:   pressure setpoint at the specified time
    \item \texttt{prf\_Lset}:   mass flow rate setpoint at the specified time
\end{itemize}
If a profile for a station includes a single entry, the quantity is held constant
at the specified value for the whole simulation. The table of the profiles for
Injection stations is fully specified by the following SQL code:
\begin{minted}{sql}
create table profiles_injection_w (
    s_number    INTEGER,
    prf_time    REAL DEFAULT 0.0 NOT NULL,
    prf_Pset    REAL DEFAULT 0.0 NOT NULL,
    prf_Lset    REAL DEFAULT 0.0 NOT NULL,

    FOREIGN KEY (s_number)
        REFERENCES stations(s_number)
);
\end{minted}

%%%%%%%%%%%%%%%%%%%%%%%%%%%%%%%%%%%%%%%%%%%%%%%%%%%%%%%%%%%%%%%%%%%%%%%%%%%%%%%%
\subsubsection{Limits and profiles: Consumption station}

Each station of type Consumption has an associated set of user-specified limits
on pressure and mass flow rate. Limits are specified per station by NDF entries
including:
\begin{itemize}
    \item \texttt{s\_number}:   number of the station
    \item \texttt{lim\_Lmin}:   minimum allowed mass flow rate
    \item \texttt{lim\_Lmax}:   maximum allowed mass flow rate
    \item \texttt{lim\_Pmin}:   minimum allowed pressure
    \item \texttt{lim\_Pmax}:   maximum allowed pressure
\end{itemize}
The table of the limits for Consumption stations is fully specified by the following SQL code:
\begin{minted}{sql}
create table limits_conspoint_wo (
    s_number    INTEGER UNIQUE,
    lim_Lmin    REAL DEFAULT 0.0 NOT NULL,
    lim_Lmax    REAL DEFAULT 0.0 NOT NULL,
    lim_Pmin    REAL DEFAULT 0.0 NOT NULL,
    lim_Pmax    REAL DEFAULT 0.0 NOT NULL,

    FOREIGN KEY (s_number)
        REFERENCES stations(s_number)
);
\end{minted}
In addition to limits, the NDFs allow to store also profiles for the quantity
controlled by the station. A mass flow rate profile is specified by triples including:
\begin{itemize}
    \item \texttt{s\_number}:   number of the station
    \item \texttt{prf\_time}:   relative time of the sample (seconds)
    \item \texttt{prf\_Lset}:   mass flow rate setpoint at the specified time
\end{itemize}
If a profile for a station includes a single entry, the quantity is held constant
at the specified value for the whole simulation. The table of the profiles for
Consumption stations is fully specified by the following SQL code:
\begin{minted}{sql}
create table profiles_conspoint_wo (
    s_number    INTEGER,
    prf_time    REAL DEFAULT 0.0 NOT NULL,
    prf_Lset    REAL DEFAULT 0.0 NOT NULL,

    CHECK(prf_Lset >= 0),

    FOREIGN KEY (s_number)
        REFERENCES stations(s_number)
);
\end{minted}

%%%%%%%%%%%%%%%%%%%%%%%%%%%%%%%%%%%%%%%%%%%%%%%%%%%%%%%%%%%%%%%%%%%%%%%%%%%%%%%%
%%%%%%%%%%%%%%%%%%%%%%%%%%%%%%%%%%%%%%%%%%%%%%%%%%%%%%%%%%%%%%%%%%%%%%%%%%%%%%%%
\subsection{Database schema: pipelines and non-pipe elements}
%%%%%%%%%%%%%%%%%%%%%%%%%%%%%%%%%%%%%%%%%%%%%%%%%%%%%%%%%%%%%%%%%%%%%%%%%%%%%%%%
\subsubsection{Pipeline types}
All the network elements with an inlet and an outled are termed ``pipeline elements''
inside Shimmer++. As for the stations, Shimmer++ handles different network elements
each uniquely identified by an integer. Pipeline types are contained in a table
fully specified by the following SQL code:
\begin{minted}{sql}
create table pipeline_types (
    p_type      INTEGER,
    t_name      TEXT NOT NULL,
    PRIMARY KEY (p_type)
);
\end{minted}

As for the stations, the above SQL table is pre-populated with the four types
of stations supported by Shimmer++, summarized in Table~\ref{tbl:pipetypes}.
\begin{table}[ht]
    \centering
    \begin{tabular}{c|c}
        \texttt{p\_type} & \texttt{t\_name} \\
        \hline
        0 & Plain pipe \\
        1 & Compressor \\
        2 & Reduction station \\
        3 & Valve \\
    \end{tabular}
    \caption{Pipe-like elements supported by Shimmer++}
    \label{tbl:pipetypes}
\end{table}

Each pipeline-like element is uniquely identified by a triple including the pipe
name \texttt{p\_name}, the originating station \texttt{s\_from} and the destination
station \texttt{s\_to}. In addition, the type of the pipeline element is stored
in the \texttt{p\_type} attribute. 
\begin{minted}{sql}
create table pipelines (
    p_name      TEXT NOT NULL,
    s_from      INTEGER NOT NULL,
    s_to        INTEGER NOT NULL,    
    p_type      INTEGER NOT NULL,
    PRIMARY KEY (p_name, s_from, s_to),

    -- The source station must exist
    FOREIGN KEY (s_from)
        REFERENCES stations(s_number),
    -- The destination station must exist
    FOREIGN KEY (s_to)
        REFERENCES stations(s_number),
    -- The pipeline type must be valid
    FOREIGN KEY (p_type)
        REFERENCES pipeline_types(p_type)
);
\end{minted}
%%%%%%%%%%%%%%%%%%%%%%%%%%%%%%%%%%%%%%%%%%%%%%%%%%%%%%%%%%%%%%%%%%%%%%%%%%%%%%%%
\subsubsection{Plain pipes}
Plain pipes are uniquely identified by a triple including the pipe
name \texttt{p\_name}, the originating station \texttt{s\_from} and the destination
station \texttt{s\_to}. Additional attributes of a pipe are its diameter,
its length, its roughness and the number of segments in which it will be split
if mesh refinement is used for quality tracking. The table where the pipes and
related parameters are stored is fully specified by the following SQL code:
\begin{minted}{sql}
create table pipe_parameters (
    p_name      TEXT NOT NULL,
    s_from      INTEGER,
    s_to        INTEGER,
    diameter    REAL DEFAULT 0.0,
    length      REAL DEFAULT 0.0,
    roughness   REAL DEFAULT 0.0,
    ref_nsegs   INTEGER DEFAULT 0,

    -- The referenced pipeline must exist
    FOREIGN KEY (p_name, s_from, s_to)
        REFERENCES pipelines(p_name, s_from, s_to),

    CHECK(ref_nsegs >= 0)
);
\end{minted}
%%%%%%%%%%%%%%%%%%%%%%%%%%%%%%%%%%%%%%%%%%%%%%%%%%%%%%%%%%%%%%%%%%%%%%%%%%%%%%%%
\subsubsection{Compressor stations}
Compressors are uniquely identified by a triple including the pipe
name \texttt{p\_name}, the originating station \texttt{s\_from} and the destination
station \texttt{s\_to}. Each compressor has an associated set of limits and an
associated temporal profile.

TODO DENERG: describe a bit the engineering meaning of each parameter.

The table containing the compressor limits is fully specified by the following
SQL code:
\begin{minted}{sql}
create table compressor_limits (
    p_name          TEXT NOT NULL,
    s_from          INTEGER,
    s_to            INTEGER,

    max_power       REAL DEFAULT 0.0,
    max_outpress    REAL DEFAULT 0.0,
    min_inpress     REAL DEFAULT 0.0,
    max_ratio       REAL DEFAULT 0.0,
    min_ratio       REAL DEFAULT 0.0,
    max_massflow    REAL DEFAULT 0.0,

    PRIMARY KEY (p_name, s_from, s_to),

    -- The referenced pipeline must exist
    FOREIGN KEY (p_name, s_from, s_to)
        REFERENCES pipelines(p_name, s_from, s_to)
);
\end{minted}

Each compressor has also an associated temporal profile.
\begin{minted}{sql}
create table compressor_profile (
    p_name      TEXT NOT NULL,
    s_from      INTEGER,
    s_to        INTEGER,

    prf_time    REAL DEFAULT 0.0,
    controlmode INTEGER DEFAULT 10, -- default OFF BYPASS
    power       REAL DEFAULT 0.0,
    outpress    REAL DEFAULT 0.0,
    inpress     REAL DEFAULT 0.0,
    ratio       REAL DEFAULT 0.0,
    massflow    REAL DEFAULT 0.0,

    -- The referenced pipeline must exist
    FOREIGN KEY (p_name, s_from, s_to)
        REFERENCES pipelines(p_name, s_from, s_to)
);
\end{minted}


%%%%%%%%%%%%%%%%%%%%%%%%%%%%%%%%%%%%%%%%%%%%%%%%%%%%%%%%%%%%%%%%%%%%%%%%%%%%%%%%
\subsubsection{Valves}
\begin{minted}{sql}
create table valve_profile (
    p_name      TEXT NOT NULL,
    s_from      INTEGER,
    s_to        INTEGER,

    prf_time    REAL DEFAULT 0.0,
    controlmode INTEGER DEFAULT 10, -- default OFF BYPASS

    -- The referenced pipeline must exist
    FOREIGN KEY (p_name, s_from, s_to)
        REFERENCES pipelines(p_name, s_from, s_to)
);
\end{minted}

%%%%%%%%%%%%%%%%%%%%%%%%%%%%%%%%%%%%%%%%%%%%%%%%%%%%%%%%%%%%%%%%%%%%%%%%%%%%%%%%
%%%%%%%%%%%%%%%%%%%%%%%%%%%%%%%%%%%%%%%%%%%%%%%%%%%%%%%%%%%%%%%%%%%%%%%%%%%%%%%%
\subsection{Database schema: network initial conditions}

To start a simulation the user must provide the initial conditions of the network,
in particular the conditions at each node and in each pipe.

Initial conditions for stations are specified in the table \texttt{station\_initial\_conditions}
as specified by the following SQL code:
\begin{minted}{sql}
create table station_initial_conditions (
    s_number    INTEGER UNIQUE NOT NULL,
    init_P      REAL DEFAULT 0.0 NOT NULL,
    init_L      REAL DEFAULT 0.0 NOT NULL,

    FOREIGN KEY (s_number)
        REFERENCES stations(s_number)
);
\end{minted}
For each station an entry composed of the station number, station pressure and
station mass flow must be provided.

Similarly to stations, also pipe initial conditions must be provided as entries
in the \texttt{pipe\_initial\_conditions} table. Each entry has to include the
station name, the start and end node and the initial value of G, as specified
by the following SQL code.
\begin{minted}{sql}
create table pipe_initial_conditions (
    p_name      TEXT NOT NULL,
    s_from      INTEGER,
    s_to        INTEGER,
    init_G      REAL DEFAULT 0.0 NOT NULL,
    
    PRIMARY KEY (p_name, s_from, s_to),

    FOREIGN KEY (p_name, s_from, s_to)
        REFERENCES pipelines(p_name, s_from, s_to)
);
\end{minted}
%%%%%%%%%%%%%%%%%%%%%%%%%%%%%%%%%%%%%%%%%%%%%%%%%%%%%%%%%%%%%%%%%%%%%%%%%%%%%%%%
%%%%%%%%%%%%%%%%%%%%%%%%%%%%%%%%%%%%%%%%%%%%%%%%%%%%%%%%%%%%%%%%%%%%%%%%%%%%%%%%
\subsection{Database schema: simulation outputs}

\begin{minted}{sql}
create table solution_station_pressures (
    s_number        INTEGER NOT NULL,
    timestep        INTEGER NOT NULL,
    pressure        REAL DEFAULT 0.0 NOT NULL,
    
    FOREIGN KEY (s_number)
        REFERENCES stations(s_number)
);
\end{minted}

\begin{minted}{sql}
create table solution_pipe_flowrates (
    p_name      TEXT NOT NULL,
    s_from      INTEGER NOT NULL,
    s_to        INTEGER NOT NULL,
    timestep    INTEGER NOT NULL,
    flowrate    REAL DEFAULT 0.0 NOT NULL,

    FOREIGN KEY (p_name, s_from, s_to)
        REFERENCES pipelines(p_name, s_from, s_to)
);
\end{minted}

\begin{minted}{sql}
create table solution_station_flowrates (
    s_number        INTEGER NOT NULL,
    timestep        INTEGER NOT NULL,
    pressure        REAL DEFAULT 0.0 NOT NULL,
    
    FOREIGN KEY (s_number)
        REFERENCES stations(s_number)
);
\end{minted}

\begin{minted}{sql}
create table solution_pipe_velocities (
    p_name      TEXT NOT NULL,
    s_from      INTEGER NOT NULL,
    s_to        INTEGER NOT NULL,
    timestep    INTEGER NOT NULL,
    velocity    REAL DEFAULT 0.0 NOT NULL,

    FOREIGN KEY (p_name, s_from, s_to)
        REFERENCES pipelines(p_name, s_from, s_to)
);
\end{minted}


