\documentclass[10pt]{article}

\usepackage[utf8]{inputenc}
\usepackage{minted}
\usepackage{fullpage}
\usepackage{hyperref}
\usepackage{newpxtext}

\title{Shimmer++ Handbook}

\begin{document}
\maketitle

\section{Model description}
TODO DENERG

\section{Network Data Files}
The network data files (NDFs) of Shimmer++ are based on SQLite, which is the
most widely deployed relational database system. This choice guarantees total
compatibility with any programming language/environment and easy manipulation of
the network data. On top of the raw SQLite database, Shimmer++ provides a C++
API and a Matlab API specific for the manipulation of the NDFs.

\subsection{Database schema}
In the following the NDF database schema will be described. The knowledge of
the database schema is needed to directly modify the NDFs via tools like
SQLite Browser (\url{https://sqlitebrowser.org/}) or to implement third party
tools able to read and write Shimmer++ NDFs.

\subsubsection{Station types}
Shimmer++ handles gas networks with different types of stations, as described
in Section~\ref{sec:shimmer_model}. Each station type is assigned a specific
numeric identifier in order to be appropriately handled by the code. Currently
implemented station types are visible in \texttt{network\_elements.hpp} and
summarized in the following code snippet: 

\begin{minted}{c++}
enum class station_type : int {
    ENTRY_P_REG = 1,            /* ReMi w/o backflow */
    ENTRY_L_REG = 2,            /* Injection w/ pressure control */
    EXIT_L_REG  = 3,            /* Consumption point w/o pressure control */
    JUNCTION    = 4,            /* Junction */
    PRIVATE_INLET  = 10,        /* Inlet, internal use only */
    PRIVATE_OUTLET = 11,        /* Outlet, internal use only */
    FICTITIOUS_JUNCTION = 100   /* For quality tracking, should not appear in DB*/
};
\end{minted}

On the NDF side, the table containing the station types is named
\texttt{station\_types} and is specified by the following SQL statement:
\begin{minted}{sql}
create table station_types (
    t_type          INTEGER,        -- Station type, as per enum above
    t_descr         TEXT NOT NULL,  -- Free-form description of the type
    t_limits_table  TEXT,           -- Table with type-specific operational limits
    t_profile_table TEXT,           -- Table with type-specific operational profiles

    PRIMARY KEY(t_type)
);
\end{minted}

\begin{itemize}
    
\end{itemize}

\subsubsection{Stations}

\begin{minted}{sql}
create table stations (
    s_number    INTEGER,
    s_name      TEXT NOT NULL,
    t_type      INTEGER,

    s_height    REAL DEFAULT 0.0 NOT NULL,
    s_latitude  REAL DEFAULT 0.0 NOT NULL,
    s_longitude REAL DEFAULT 0.0 NOT NULL,

    PRIMARY KEY(s_number),
    
    -- The type of the station must be well-defined
    FOREIGN KEY (t_type)
        REFERENCES station_types(t_type),
    
    CHECK(s_number >= 0)
);
\end{minted}

\subsubsection{Limits and profiles - ReMi station}

\begin{minted}{sql}
-- Limits
--  s_number:   number of the station
--  lim_Lmin:   minimum allowed mass flow rate
--  lim_Lmax:   maximum allowed mass flow rate
--  lim_Pmin:   minimum allowed pressure
--  lim_Pmax:   maximum allowed pressure

create table limits_remi_wo (
    s_number    INTEGER UNIQUE,
    lim_Lmin    REAL DEFAULT 0.0 NOT NULL,
    lim_Lmax    REAL DEFAULT 0.0 NOT NULL,
    lim_Pmin    REAL DEFAULT 0.0 NOT NULL,
    lim_Pmax    REAL DEFAULT 0.0 NOT NULL,

    FOREIGN KEY (s_number)
        REFERENCES stations(s_number)
);

-- Profiles
--  s_number:   number of the station
--  prf_time:   relative time of the sample
--  prf_Pset:   pressure setpoint at the specified time

create table profiles_remi_wo (
    s_number    INTEGER,
    prf_time    REAL DEFAULT 0.0 NOT NULL,
    prf_Pset    REAL DEFAULT 0.0 NOT NULL,

    FOREIGN KEY (s_number)
        REFERENCES stations(s_number)
);
\end{minted}

\subsubsection{Limits and profiles - Injection station}

\begin{minted}{sql}
-- Limits
--  s_number:   number of the station
--  lim_Lmin:   minimum allowed mass flow rate
--  lim_Lmax:   maximum allowed mass flow rate
--  lim_Pmin:   minimum allowed pressure
--  lim_Pmax:   maximum allowed pressure
--  parm_f:     scale factor named "f" in the slides

create table limits_injection_w (
    s_number    INTEGER UNIQUE,
    lim_Lmin    REAL DEFAULT 0.0 NOT NULL,
    lim_Lmax    REAL DEFAULT 0.0 NOT NULL,
    lim_Pmin    REAL DEFAULT 0.0 NOT NULL,
    lim_Pmax    REAL DEFAULT 0.0 NOT NULL,
    parm_f      REAL DEFAULT 1.0 NOT NULL,

    FOREIGN KEY (s_number)
        REFERENCES stations(s_number)
);

-- Profiles
--  s_number:   number of the station
--  prf_time:   relative time of the sample
--  prf_Lset:   mass flow rate setpoint at the specified time

create table profiles_injection_w (
    s_number    INTEGER,
    prf_time    REAL DEFAULT 0.0 NOT NULL,
    prf_Pset    REAL DEFAULT 0.0 NOT NULL,
    prf_Lset    REAL DEFAULT 0.0 NOT NULL,

    FOREIGN KEY (s_number)
        REFERENCES stations(s_number)
);
\end{minted}

\subsubsection{Limits and profiles - Consumption station}

\begin{minted}{sql}
-- Limits
--  s_number:   number of the station
--  lim_Lmin:   minimum allowed mass flow rate
--  lim_Lmax:   maximum allowed mass flow rate
--  lim_Pmin:   minimum allowed pressure
--  lim_Pmax:   maximum allowed pressure

create table limits_conspoint_wo (
    s_number    INTEGER UNIQUE,
    lim_Lmin    REAL DEFAULT 0.0 NOT NULL,
    lim_Lmax    REAL DEFAULT 0.0 NOT NULL,
    lim_Pmin    REAL DEFAULT 0.0 NOT NULL,
    lim_Pmax    REAL DEFAULT 0.0 NOT NULL,

    FOREIGN KEY (s_number)
        REFERENCES stations(s_number)
);

-- Profiles
--  s_number:   number of the station
--  prf_time:   relative time of the sample
--  prf_Lset:   pressure setpoint at the specified time

create table profiles_conspoint_wo (
    s_number    INTEGER,
    prf_time    REAL DEFAULT 0.0 NOT NULL,
    prf_Lset    REAL DEFAULT 0.0 NOT NULL,

    CHECK(prf_Lset >= 0),

    FOREIGN KEY (s_number)
        REFERENCES stations(s_number)
);
\end{minted}

\end{document}
