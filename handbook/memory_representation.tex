\section{In memory representation}
Network modelling can be mathematically abstracted into a graph, simplified into nodes and 1D elements, representing the pipes, and their interaction. The neywork graph is implemented using Boost Graph Library, which provides a labeled graph implementation and related algorithms. The network data in particular is stored into the graph labels. Bearing in mind, that pipes are inherently not directional, in contrast to physical variables as velocity or flux. The graph is then set as an \cred{undirected graph} with values in \cpipes{pipes}/\cnodes{nodes} assigned via the graph properties. The definition of the graph type is given below
\begin{verbatim}
    using infrastructure_graph = adjacency_list< listS, vecS,
                                    undirectedS, vertex_properties, edge_properties>;    
\end{verbatim}    
where the \texttt{infrastructure\_graph} type is already defined in \texttt{shimmer++/src/infrastructure\_graph.h} 
\begin{minted}[linenos=true,numbersep=1pt,frame=lines,framesep=0.5mm]{cpp}
 using namespace boost;

 using infrastructure_graph = adjacency_list< listS, vecS,
                                    undirectedS, vertex_properties, edge_properties>;
 using sptr_pipe_t = std::shared_ptr<edge_station::station>; 
 using uptr_node_t = std::unique_ptr<station>;
 
 struct vertex_properties                       struct edge_properties 
 {                                              {
    int             u_snum;                         pipe_type   type;
    int             i_snum;                         int         branch_num;
    station_type    type;                           double      length;
    double          height;                         double      diameter;
    double          latitude;                       double      friction_factor;
    double          longitude;                      sptr_pipe_t pipe_station;
    vector_t        gas_mixture;                    std::string name;
    uptr_node_t     node_station;                   int         i_sfrom;
                                                    int         i_sto;            
 };                                             };  

enum class pipe_type = { pipe, resistor, compressor, regulator,  valve};
\end{minted}

    
\begin{tikzpicture}[remember picture,overlay]
    \def\a{-0.5}
    \draw[dpipe] ( 8.0, 2+\a) -- ( 8.0, 7.00) -- (14.5, 7.00) -- (14.5, 2+\a) -- cycle;
    \draw[dnode] (-0.5, 2+\a) -- (-0.5, 7.00) -- ( 6  , 7.00) -- ( 6 , 2+\a) -- cycle;
\end{tikzpicture}

\textbf{Hands-on:}
Network graphs are usually populated via network data files. However, due to testing purposes  the data structures can still be populated manually. For the first option see Section \ref{SectionDatabase}. In the second case, you are asked to introduce pipe and node properties one by one as will explained in what follows. Lets take as example the graph in Fig. \ref{fig: graph_example} of a simplified  gas network. 

\begin{figure}[H]
    \centering
            % Graphic for TeX using PGF
% Title: /home/karol/Documents/UNIVERSITA/POLITO/PRESENTATIONS/SHIMMER_2024_01/img_code/pipe_network.dia
% Creator: Dia v0.97.3
% CreationDate: Mon Jan 22 14:43:56 2024
% For: karol
% \usepackage{tikz}
% The following commands are not supported in PSTricks at present
% We define them conditionally, so when they are implemented,
% this pgf file will use them.
\ifx\du\undefined
  \newlength{\du}
\fi
\setlength{\du}{11\unitlength}
\begin{tikzpicture}[even odd rule]
\pgftransformxscale{1.000000}
\pgftransformyscale{-1.000000}
\definecolor{dialinecolor}{rgb}{0.000000, 0.000000, 0.000000}
\pgfsetstrokecolor{dnode}
\pgfsetstrokeopacity{1.000000}
\definecolor{diafillcolor}{rgb}{1.000000, 1.000000, 1.000000}
\pgfsetfillcolor{diafillcolor}
\pgfsetfillopacity{1.000000}
\pgfsetlinewidth{0.100000\du}
\pgfsetdash{}{0pt}
\pgfsetmiterjoin
\definecolor{diafillcolor}{rgb}{1.000000, 1.000000, 1.000000}
\pgfsetfillcolor{diafillcolor}
\pgfsetfillopacity{1.000000}
\pgfpathellipse{\pgfpoint{8.693572\du}{5.676288\du}}{\pgfpoint{0.693572\du}{0\du}}{\pgfpoint{0\du}{0.676288\du}}
\pgfusepath{fill}
\definecolor{dialinecolor}{rgb}{0.937255, 0.000000, 0.701961}
\pgfsetstrokecolor{dnode}
\pgfsetstrokeopacity{1.000000}
\pgfpathellipse{\pgfpoint{8.693572\du}{5.676288\du}}{\pgfpoint{0.693572\du}{0\du}}{\pgfpoint{0\du}{0.676288\du}}
\pgfusepath{stroke}
% setfont left to latex
% setfont left to latex
\definecolor{dialinecolor}{rgb}{0.000000, 0.000000, 0.000000}
\pgfsetstrokecolor{dnode}
\pgfsetstrokeopacity{1.000000}
\definecolor{diafillcolor}{rgb}{0.000000, 0.000000, 0.000000}
\pgfsetfillcolor{diafillcolor}
\pgfsetfillopacity{1.000000}
\node[anchor=base,inner sep=0pt, outer sep=0pt,color=dnode] at (8.693572\du,5.961288\du){0};
\pgfsetlinewidth{0.100000\du}
\pgfsetdash{}{0pt}
\pgfsetmiterjoin
\definecolor{diafillcolor}{rgb}{1.000000, 1.000000, 1.000000}
\pgfsetfillcolor{diafillcolor}
\pgfsetfillopacity{1.000000}
\pgfpathellipse{\pgfpoint{8.693572\du}{9.676288\du}}{\pgfpoint{0.693572\du}{0\du}}{\pgfpoint{0\du}{0.676288\du}}
\pgfusepath{fill}
\definecolor{dialinecolor}{rgb}{0.937255, 0.000000, 0.701961}
\pgfsetstrokecolor{dnode}
\pgfsetstrokeopacity{1.000000}
\pgfpathellipse{\pgfpoint{8.693572\du}{9.676288\du}}{\pgfpoint{0.693572\du}{0\du}}{\pgfpoint{0\du}{0.676288\du}}
\pgfusepath{stroke}
% setfont left to latex
% setfont left to latex
\definecolor{dialinecolor}{rgb}{0.000000, 0.000000, 0.000000}
\pgfsetstrokecolor{dialinecolor}
\pgfsetstrokeopacity{1.000000}
\definecolor{diafillcolor}{rgb}{0.000000, 0.000000, 0.000000}
\pgfsetfillcolor{diafillcolor}
\pgfsetfillopacity{1.000000}
\node[anchor=base,inner sep=0pt, outer sep=0pt,color=dnode] at (8.693572\du,9.961288\du){1};
\pgfsetlinewidth{0.100000\du}
\pgfsetdash{}{0pt}
\pgfsetmiterjoin
\definecolor{diafillcolor}{rgb}{1.000000, 1.000000, 1.000000}
\pgfsetfillcolor{diafillcolor}
\pgfsetfillopacity{1.000000}
\pgfpathellipse{\pgfpoint{6.753572\du}{13.676288\du}}{\pgfpoint{0.753572\du}{0\du}}{\pgfpoint{0\du}{0.676288\du}}
\pgfusepath{fill}
\definecolor{dialinecolor}{rgb}{0.937255, 0.000000, 0.701961}
\pgfsetstrokecolor{dnode}
\pgfsetstrokeopacity{1.000000}
\pgfpathellipse{\pgfpoint{6.753572\du}{13.676288\du}}{\pgfpoint{0.753572\du}{0\du}}{\pgfpoint{0\du}{0.676288\du}}
\pgfusepath{stroke}
% setfont left to latex
% setfont left to latex
\definecolor{dialinecolor}{rgb}{0.000000, 0.000000, 0.000000}
\pgfsetstrokecolor{dialinecolor}
\pgfsetstrokeopacity{1.000000}
\definecolor{diafillcolor}{rgb}{0.000000, 0.000000, 0.000000}
\pgfsetfillcolor{diafillcolor}
\pgfsetfillopacity{1.000000}
\node[anchor=base,inner sep=0pt, outer sep=0pt,color=dnode] at (6.753572\du,13.961288\du){2};
\pgfsetlinewidth{0.100000\du}
\pgfsetdash{}{0pt}
\pgfsetmiterjoin
\definecolor{diafillcolor}{rgb}{1.000000, 1.000000, 1.000000}
\pgfsetfillcolor{diafillcolor}
\pgfsetfillopacity{1.000000}
\pgfpathellipse{\pgfpoint{14.693572\du}{13.676288\du}}{\pgfpoint{0.693572\du}{0\du}}{\pgfpoint{0\du}{0.676288\du}}
\pgfusepath{fill}
\definecolor{dialinecolor}{rgb}{0.937255, 0.000000, 0.701961}
\pgfsetstrokecolor{dnode}
\pgfsetstrokeopacity{1.000000}
\pgfpathellipse{\pgfpoint{14.693572\du}{13.676288\du}}{\pgfpoint{0.693572\du}{0\du}}{\pgfpoint{0\du}{0.676288\du}}
\pgfusepath{stroke}
% setfont left to latex
% setfont left to latex
\definecolor{dialinecolor}{rgb}{0.000000, 0.000000, 0.000000}
\pgfsetstrokecolor{dialinecolor}
\pgfsetstrokeopacity{1.000000}
\definecolor{diafillcolor}{rgb}{0.000000, 0.000000, 0.000000}
\pgfsetfillcolor{diafillcolor}
\pgfsetfillopacity{1.000000}
\node[anchor=base,inner sep=0pt, outer sep=0pt,color=dnode] at (14.693572\du,13.961288\du){4};
\pgfsetlinewidth{0.100000\du}
\pgfsetdash{}{0pt}
\pgfsetmiterjoin
\definecolor{diafillcolor}{rgb}{1.000000, 1.000000, 1.000000}
\pgfsetfillcolor{diafillcolor}
\pgfsetfillopacity{1.000000}
\pgfpathellipse{\pgfpoint{10.693572\du}{13.676288\du}}{\pgfpoint{0.693572\du}{0\du}}{\pgfpoint{0\du}{0.676288\du}}
\pgfusepath{fill}
\definecolor{dialinecolor}{rgb}{0.937255, 0.000000, 0.701961}
\pgfsetstrokecolor{dnode}
\pgfsetstrokeopacity{1.000000}
\pgfpathellipse{\pgfpoint{10.693572\du}{13.676288\du}}{\pgfpoint{0.693572\du}{0\du}}{\pgfpoint{0\du}{0.676288\du}}
\pgfusepath{stroke}
% setfont left to latex
% setfont left to latex
\definecolor{dialinecolor}{rgb}{0.000000, 0.000000, 0.000000}
\pgfsetstrokecolor{dialinecolor}
\pgfsetstrokeopacity{1.000000}
\definecolor{diafillcolor}{rgb}{0.000000, 0.000000, 0.000000}
\pgfsetfillcolor{diafillcolor}
\pgfsetfillopacity{1.000000}
\node[anchor=base,inner sep=0pt, outer sep=0pt,color=dnode] at (10.693572\du,13.961288\du){3};
\pgfsetlinewidth{0.050000\du}
\pgfsetdash{}{0pt}
\pgfsetbuttcap
{
\definecolor{diafillcolor}{rgb}{0.000000, 0.501961, 0.501961}
\pgfsetfillcolor{diafillcolor}
\pgfsetfillopacity{1.000000}
% was here!!!
\definecolor{dialinecolor}{rgb}{0.000000, 0.501961, 0.501961}
\pgfsetstrokecolor{dpipe}
\pgfsetstrokeopacity{1.000000}
\draw (8.693572\du,6.351417\du)--(8.693570\du,9.000000\du);
}
\pgfsetlinewidth{0.050000\du}
\pgfsetdash{}{0pt}
\pgfsetbuttcap
{
\definecolor{diafillcolor}{rgb}{0.000000, 0.501961, 0.501961}
\pgfsetfillcolor{diafillcolor}
\pgfsetfillopacity{1.000000}
% was here!!!
\definecolor{dialinecolor}{rgb}{0.000000, 0.501961, 0.501961}
\pgfsetstrokecolor{dpipe}
\pgfsetstrokeopacity{1.000000}
\draw (8.693570\du,10.352600\du)--(6.753570\du,13.000000\du);
}
\pgfsetlinewidth{0.050000\du}
\pgfsetdash{}{0pt}
\pgfsetbuttcap
{
\definecolor{diafillcolor}{rgb}{0.000000, 0.501961, 0.501961}
\pgfsetfillcolor{diafillcolor}
\pgfsetfillopacity{1.000000}
% was here!!!
\definecolor{dialinecolor}{rgb}{0.000000, 0.501961, 0.501961}
\pgfsetstrokecolor{dpipe}
\pgfsetstrokeopacity{1.000000}
\draw (8.693570\du,10.352600\du)--(10.693600\du,13.000000\du);
}
\pgfsetlinewidth{0.050000\du}
\pgfsetdash{}{0pt}
\pgfsetbuttcap
{
\definecolor{diafillcolor}{rgb}{0.000000, 0.501961, 0.501961}
\pgfsetfillcolor{diafillcolor}
\pgfsetfillopacity{1.000000}
% was here!!!
\definecolor{dialinecolor}{rgb}{0.000000, 0.501961, 0.501961}
\pgfsetstrokecolor{dpipe}
\pgfsetstrokeopacity{1.000000}
\draw (10.000000\du,13.676300\du)--(7.507140\du,13.676300\du);
}
\pgfsetlinewidth{0.050000\du}
\pgfsetdash{}{0pt}
\pgfsetbuttcap
{
\definecolor{diafillcolor}{rgb}{0.000000, 0.501961, 0.501961}
\pgfsetfillcolor{diafillcolor}
\pgfsetfillopacity{1.000000}
% was here!!!
\definecolor{dialinecolor}{rgb}{0.000000, 0.501961, 0.501961}
\pgfsetstrokecolor{dpipe}
\pgfsetstrokeopacity{1.000000}
\draw (11.387100\du,13.676300\du)--(14.000000\du,13.676300\du);
}
% setfont left to latex
% setfont left to latex
\definecolor{dialinecolor}{rgb}{0.000000, 0.501961, 0.501961}
\pgfsetstrokecolor{dialinecolor}
\pgfsetstrokeopacity{1.000000}
\definecolor{diafillcolor}{rgb}{0.000000, 0.501961, 0.501961}
\pgfsetfillcolor{diafillcolor}
\pgfsetfillopacity{1.000000}
\node[anchor=base west,inner sep=0pt,outer sep=0pt,color=dpipe] at (7.5000000\du,8.000000\du){0};
% setfont left to latex
% setfont left to latex
\definecolor{dialinecolor}{rgb}{0.000000, 0.501961, 0.501961}
\pgfsetstrokecolor{dialinecolor}
\pgfsetstrokeopacity{1.000000}
\definecolor{diafillcolor}{rgb}{0.000000, 0.501961, 0.501961}
\pgfsetfillcolor{diafillcolor}
\pgfsetfillopacity{1.000000}
\node[anchor=base west,inner sep=0pt,outer sep=0pt,color=dpipe] at (6.5000000\du,12.000000\du){3};
% setfont left to latex
% setfont left to latex
\definecolor{dialinecolor}{rgb}{0.000000, 0.501961, 0.501961}
\pgfsetstrokecolor{dialinecolor}
\pgfsetstrokeopacity{1.000000}
\definecolor{diafillcolor}{rgb}{0.000000, 0.501961, 0.501961}
\pgfsetfillcolor{diafillcolor}
\pgfsetfillopacity{1.000000}
\node[anchor=base west,inner sep=0pt,outer sep=0pt,color=dpipe] at (10.5000000\du,12.000000\du){1};
% setfont left to latex
% setfont left to latex
\definecolor{dialinecolor}{rgb}{0.000000, 0.501961, 0.501961}
\pgfsetstrokecolor{dialinecolor}
\pgfsetstrokeopacity{1.000000}
\definecolor{diafillcolor}{rgb}{0.000000, 0.501961, 0.501961}
\pgfsetfillcolor{diafillcolor}
\pgfsetfillopacity{1.000000}
\node[anchor=base west,inner sep=0pt,outer sep=0pt,color=dpipe] at (8.450000\du,14.900000\du){2};
% setfont left to latex
% setfont left to latex
\definecolor{dialinecolor}{rgb}{0.000000, 0.501961, 0.501961}
\pgfsetstrokecolor{dialinecolor}
\pgfsetstrokeopacity{1.000000}
\definecolor{diafillcolor}{rgb}{0.000000, 0.501961, 0.501961}
\pgfsetfillcolor{diafillcolor}
\pgfsetfillopacity{1.000000}
\node[anchor=base west,inner sep=0pt,outer sep=0pt,color=dpipe] at (12.590000\du,14.895000\du){4};
\end{tikzpicture}
    
    \caption{Graph representation of a simplified gas network.}
    \label{fig: graph_example}
\end{figure}

\subsection{Define the graph}
In the main function add the definition of the graph. For the sake of clearness, let us proceed step by step and introduce two functions aimed to add first the vertex, \texttt{make\_init\_vertex} and later on the pipes properties, \texttt{make\_init\_pipes}. 

\begin{minted}[linenos=true,numbersep=5pt,frame=lines,framesep=0.5mm ]{cpp}
    int main()
    {
        infrastructure_graph igraph;
        std::vector<vertex_descriptor> vds;
        make_init_vertex(igraph, vds);
        make_init_pipes(igraph, vds);
    
        return 0; 
    }
\end{minted}   


\subsection{Add nodes specification}
Let us suppose that the data shown in Tab. \ref{tab: node_data} is attached to the network vertices. For this example, no specification about mixture composition is  taken into account. Nonetheless, this functionality is supported and can be easily added. See example \cred{...}

\begin{table}[H]
    \centering
       \begin{tabular}{ccrrrcccc}
        & & & & & & & & \\
        \hline 
        &  & \multicolumn{3}{c}{Node properties}  & & \multicolumn{3}{c}{Mixture composition} \\ \cline{3-5} \cline{7-9}     
        \multirow{5}{*}{\rotatebox[origin=c]{90}{\cnodes{Nodes}}} 
        &           & G[kg/s] &p[Pa] &H[m]  && CH4 & N2& {\dots} \\ \hline
        &\cnodes{0} &	\cellcolor{dnode!30} 5000  & \cellcolor{dnode!30}-60  &\cellcolor{dnode!30}10.0  && -& -& -\\
        &\cnodes{1} &\cellcolor{dnode!30}	0	  &\cellcolor{dnode!30}  20  &\cellcolor{dnode!30} 20.0  && -& - & {-}\\
        &\cnodes{2} &\cellcolor{dnode!30}	0	  & \cellcolor{dnode!30} 25  &\cellcolor{dnode!30} 30.0  && -& - & {-}\\ 
        &\cnodes{3} &\cellcolor{dnode!30}	0	  &\cellcolor{dnode!30}  35  &\cellcolor{dnode!30}40.0  && -& - &  {-}\\ 
        &\cnodes{4} &\cellcolor{dnode!30}	0	  &\cellcolor{dnode!30}  50  &\cellcolor{dnode!30}50.0  && -& - & {-}\\
        \hline
        &&&&&&&&\\
        \end{tabular}
    \caption{Parameters concerning node specification.}
    \label{tab: node_data}
\end{table}
 
%\begin{tikzpicture}[remember picture,overlay]
%    \draw[dnode] (9.3, 1.3) -- (9.3,2.85) -- ( 7.15, 2.85) -- (7.15, 1.3) -- cycle;
%\end{tikzpicture}

        
%\begin{tikzpicture}[remember picture,overlay]
%    \draw[dpipe](10.2, 0.8) -- (10.2, 3.10) -- (13.5, 3.10) -- (13.5, 0.8) -- cycle;
%    \draw[dnode]( 2.0, 0.8) -- (2.0, 3.10) -- ( 4.7, 3.10) -- ( 4.7, 0.8) -- cycle;
%\end{tikzpicture}

%In the following snippet code you will distinguish a lambda function called \texttt{add\_vertex}, which internally call the \texttt{boost::add\_vertex}. For the sake of simplicity, this will not be further explained, but keep in mind this is needed for the management of internal pointers in the vertex properties. 
%The vector \texttt{vds} stores the nodes created, which will be used in the following for the definition of the edges.

In the following snippet code, you will distinguish a std::vector \texttt{vds} and a lambda function called \texttt{add\_vertex} (see line 6), which internally calls the \texttt{boost::add\_vertex}. The first one is used for the storage of all nodes created at the \texttt{add\_vertex} call  and it will be used in the following for the definition of the edges. For the sake of simplicity, the lambda \texttt{add\_vertex} will not be further explained, but keep in mind it  is needed for the management of internal pointers in the vertex properties, associated to its non-pipe elements. 

\begin{minted}[linenos=true,numbersep=1pt,frame=lines,framesep=2mm]{cpp}
static void
make_init_vertex(infrastructure_graph& igraph, std::vector<vertex_descriptor>& vds)
{
    auto add_vertex = [&](vertex_properties&& vp) 
    {
        auto v = boost::add_vertex(igraph);
        igraph[v] = std::move(vp);
        return v;
    };

    // Insert station config (name, no., pressure, flux, height)
    vds.push_back( add_vertex( vertex_properties( "station 0", 0, 5000.,-60, 10. )) );
    vds.push_back( add_vertex( vertex_properties( "station 1", 1,    0., 20, 20. )) );
    vds.push_back( add_vertex( vertex_properties( "station 2", 2,    0., 25, 30. )) );
    vds.push_back( add_vertex( vertex_properties( "station 3", 3,    0., 35, 40. )) );
    vds.push_back( add_vertex( vertex_properties( "station 4", 4,    0., 50, 50. )) );
}
\end{minted}

\subsection{Add pipes specification}
Let us now continue with the \cpipes{pipes} specification using the values in Tab. \ref{tab: pipe_specification}. 


\begin{table}[H]
    \centering
            \begin{tabular}{cccccrrr}
            & & & & & & & \\
            \hline
            &  & \multicolumn{2}{c}{Nodes}&&\multicolumn{3}{c}{Pipe properties} \\ \cline{3-4} \cline{6-8}    
            &  & In	&Out	&&L[m]	&D[m]	&epsi[m] \\ \hline
            \multirow{5}{*}{\rotatebox[origin=c]{90}{\cpipes{Pipes}}}
            & \cpipes{0} &	\cnodes{0} &	\cnodes{1} &&	{\cellcolor{dpipe!30}} 80.0  &{\cellcolor{dpipe!30}}0.6	&{\cellcolor{dpipe!30}}1.2e-5 \\
            &\cpipes{1} &	\cnodes{1} &	\cnodes{3} &&{\cellcolor{dpipe!30}}	90.0  &{\cellcolor{dpipe!30}}0.5	&{\cellcolor{dpipe!30}}1.3e-5 \\
            &\cpipes{2} &	\cnodes{3} &	\cnodes{2} &&{\cellcolor{dpipe!30}}	100.0 &{\cellcolor{dpipe!30}}0.4	&{\cellcolor{dpipe!30}}1.4e-5 \\ 
            &\cpipes{3} &	\cnodes{1} &	\cnodes{2} &&{\cellcolor{dpipe!30}}	110.0 &{\cellcolor{dpipe!30}}0.3	&{\cellcolor{dpipe!30}}1.5e-5 \\ 
            &\cpipes{4} &	\cnodes{3} &	\cnodes{4} &&{\cellcolor{dpipe!30}}	120.0 &{\cellcolor{dpipe!30}}0.2	&{\cellcolor{dpipe!30}}1.6e-5 \\ \hline 
            &&&&&&& \\
            \end{tabular}
    \caption{Parameters concerning pipe specification.}
    \label{tab: pipe_specification}
\end{table}
In order to add the properties, first create an object of type \texttt{edge\_properties} instantiating it with the data related to the pipes: length, diameter and factor epsi.   Afterwards, the edges of the graph are created using the \texttt{boost::add\_edges} function while insertions of the pipes properties are given as parameters. The first two entrances (arguments) are the nodes defining the edge (two first columns in Tab. \ref{tab: pipe_specification}) followed by the object with  its corresponding properties. 

\begin{minted}[linenos=true,numbersep=1pt,frame=lines,framesep=2mm]{cpp}
static void
make_init_pipes(infrastructure_graph& igraph, std::vector<vertex_descriptor>& vds)
{
    edge_properties ep0  = {pipe_type::pipe, 0,   80, 0.6, 1.2e-5};
    edge_properties ep1  = {pipe_type::pipe, 1,   90, 0.5, 1.3e-5};
    edge_properties ep2  = {pipe_type::pipe, 2,  100, 0.4, 1.4e-5};
    edge_properties ep3  = {pipe_type::pipe, 3,  110, 0.3, 1.5e-5};
    edge_properties ep4  = {pipe_type::pipe, 4,   80, 0.2, 1.6e-5};

    boost::add_edge( vds[0], vds[ 1], ep0, igraph);
    boost::add_edge( vds[1], vds[ 3], ep1, igraph);
    boost::add_edge( vds[3], vds[ 2], ep2, igraph);
    boost::add_edge( vds[1], vds[ 2], ep3, igraph);
    boost::add_edge( vds[3], vds[ 4], ep4, igraph);
}
\end{minted}

\subsection{Incidence matrix A}
The incidence matrix, as well as the graph, carries information about the connection between nodes and pipes. Hence, it is directly built from graph and used in the linear system on section \ref{}. See also Fig. \ref{fig: graph_example}.  It is also used to exploit vectorization in some parts of the code, instead of calling the graph. The complete implementation is written in \texttt{Incidence.h}.   

\begin{table}[H]
    \centering
    \begin{tabular}{rrrrrrr}
        \hline
         &  & \multicolumn{5}{c}{\cpipes{Pipes}} \\ 
         &  &  \cpipes{0}&  \cpipes{1}&  \cpipes{2}&  \cpipes{3}&  \cpipes{4} \\ \hline
        \multirow{5}{*}{\rotatebox[origin=c]{90}{\cnodes{Nodes}}}
         & \cnodes{0}&   1&   &   &   &   \\
         & \cnodes{1}&  -1&  1&   &  1&   \\
         & \cnodes{2}&   &    & -1& -1&   \\
         & \cnodes{3}&   &  -1&  1&   & 1 \\
         & \cnodes{4}&   &    &   &   &-1 \\ \hline
    \end{tabular}
    \caption{Incidence matrix based on the graph in Fig. \ref{fig: graph_example}.}
\end{table}

\subsubsection{Hands-on:}
Recall first to create a graph with the specification, as proposed in the previous examples. For this case, an slightly modification is done to summarize all in a unique function \texttt{make\_init\_graph}. 
\begin{minted}[linenos=true,numbersep=1pt,frame=lines,framesep=2mm]{cpp}
    int main()
    {
        // Define graph
        infrastructure_graph graph;
        make_init_graph(graph);

        // Build the incidence matrix A
        incidence inc(graph);

        // Get and print (std::cout) incidene matrix
        std::cout << "Incidence A: \n" << inc.matrix() << std::endl;
        std::cout << "Incidence A (only inlet): \n" << inc.matrix_in()<< std::endl;
        std::cout << "Incidence A (only outlet): \n" << inc.matrix_out()<< std::endl;
    
        return 0;
    }
\end{minted}
